\documentclass[10pt,UTF8]{book} %% ctexart

\title{\textbf{矩阵微积分}}
\author{钱锋\thanks{Email: strik0r.qf@gmail.com}${}^,$\thanks{
    西北工业大学软件学院, School of Software, Northwestern Polytechnical University, 西安 710072
}}

\usepackage{ctex}
\usepackage{graphicx}
\usepackage[toc]{multitoc}
\usepackage{booktabs}
\usepackage{longtable}
\usepackage{amsthm, amssymb, amsmath, mathrsfs, mhchem}
\usepackage{tikz,circuitikz}
\usetikzlibrary{decorations.markings, angles, quotes}
\usetikzlibrary{shapes,arrows.meta,positioning}
\usepackage{tikz-cd}
\usepackage{pgfplots}
\usepackage{tikz-3dplot}
\usepackage{extpfeil}
\usepackage{diagbox}
\usepackage{float}
\usepackage{hyperref}
\hypersetup{hidelinks,
    colorlinks = true,
    allcolors = black,
    pdfstartview = Fit,
    breaklinks = true}
\usepackage{caption}
\usepackage{enumitem}
\usepackage{siunitx}
\usepackage{subcaption}
\usepackage{tasks}

\input{LaTeX_tem/fancyhdr_settings.tex}
\input{LaTeX_tem/titlesec_settings.tex}
\input{LaTeX_tem/geometry_settings.tex}
\input{LaTeX_tem/mdframed_settings.tex}
\input{LaTeX_tem/listings_settings.tex}

\usepackage{smartdiagram} % 表格对角线
\everymath{\displaystyle}
\usepackage{tasks}

\begin{document}
\input{LaTeX_tem/theoremstyles.tex}
% 使用 IEEE 样式
\ctikzset{logic ports=ieee}

\pagestyle{empty}
\begin{titlepage}
    \thispagestyle{empty}
    \centering
        \vspace*{3cm}
        \includegraphics[width=0.5\textwidth]{pic/npu_2.png}\par
        \vspace{1em}
        \includegraphics[width=0.5\textwidth]{pic/npu_1.png}\par
    \vspace*{1em}
        \begin{center}
            \Huge \heiti \textbf{矩阵微积分}

            Matrix Calculus
        \end{center}

        \vspace{14em}
        \begin{center}
        \songti

        \kaishu 软件学院 \, \heiti\textbf{钱锋} \quad \songti 编
        \vspace{0.5em}

    \today
    \end{center}
\end{titlepage}
\cleardoublepage
\maketitle
\cleardoublepage
\frontmatter
\newpage
\pagestyle{plain}
\makeatother

\pagenumbering{roman} % 切换回罗马数字页码
\addtocontents{toc}{\protect\thispagestyle{empty}}
\pagestyle{plain}
{\tableofcontents}
\newpage
\thispagestyle{empty}
\cleardoublepage % 确保正文从奇数页开始


% 设置章节标题页的页眉和页脚为空白页样式
\makeatletter
\let\ps@plain\ps@empty
\makeatother

\mainmatter
\chapter{概述}

处理单变量的标量值函数总是简单的.

\begin{itemize}[itemsep=0pt]
    \item 单变量函数的导数仍为单变量函数;
    \item 函数的线性化 (linearization) 具有
    \[ y-y_0 \approx f'(x_0) (x - x_0) \]
    的形式. 我们也用 $\Delta$ 来表示有限的扰动 (finite perturbation),
    因此上式也可以写为
    \[ \Delta y \approx f'(x_0)\Delta x. \]
\end{itemize}

\input{LateX_tem/appendix.tex}

\onecolumn
\begin{thebibliography}{1}
    \addcontentsline{toc}{chapter}{参考文献}

\end{thebibliography}

\input{LateX_tem/endpage.tex}

\end{document}