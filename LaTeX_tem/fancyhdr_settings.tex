\usepackage{fancyhdr} % 用于自定义页眉页脚


% 设置页眉页脚样式
\fancypagestyle{plain}{%
    \fancyhf{} % 清空页眉页脚
    \fancyhead[RO,LE]{·\thepage·} % 页眉显示页码, RO表示奇数页右侧, LE表示偶数页左侧
    \fancyhead[LO]{\nouppercase{\rightmark}} % 页眉显示小节标题, LO表示奇数页左侧
    \fancyhead[RE]{\nouppercase{\leftmark}} % 页眉显示章节标题, RE表示偶数页右侧
    \renewcommand{\headrulewidth}{0.4pt} % 设置页眉横线的宽度
    \renewcommand{\footrulewidth}{0pt} % 取消页脚横线
}

\renewcommand{\headrule}{\hrule width\textwidth height\headrulewidth\vskip-\headrulewidth}

% % 取消奇偶页的页眉偏移
% \fancyhfoffset[RO,LE]{0pt}

% % 取消奇偶页的页眉偏移
% \fancyhfoffset[RO,LE]{0pt}

% 定义取消页眉的命令
\newcommand{\cancelheader}{%
    \fancyhead{} % 清空页眉
    \renewcommand{\headrulewidth}{0pt} % 取消页眉横线
    \renewcommand{\footrulewidth}{0pt} % 设置页脚横线的宽度
}

\renewcommand{\chaptermark}[1]{\markboth{第 \thechapter 章 \hspace{1em} #1}{}} % 在章节标题前添加 "Chapter x: "
\renewcommand{\sectionmark}[1]{\markright{\thesection \, #1}} % 如果需要定义\rightmark,可以使用这行代码